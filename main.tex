\documentclass[a4paper,10pt,landscape]{article}
\usepackage{amssymb,amsmath,amsthm,amsfonts}
\usepackage{multicol,multirow}
\usepackage{calc}
\usepackage{ifthen}
\usepackage[landscape]{geometry}
\usepackage[colorlinks=true,citecolor=blue,linkcolor=blue]{hyperref}


\ifthenelse{\lengthtest { \paperwidth = 11in}}
    { \geometry{top=.5in,left=.5in,right=.5in,bottom=.5in} }
	{\ifthenelse{ \lengthtest{ \paperwidth = 297mm}}
		{\geometry{top=1cm,left=1cm,right=1cm,bottom=1cm} }
		{\geometry{top=1cm,left=1cm,right=1cm,bottom=1cm} }
	}
\pagestyle{empty}
\makeatletter
\renewcommand{\section}{\@startsection{section}{1}{0mm}%
                                {-1ex plus -.5ex minus -.2ex}%
                                {0.5ex plus .2ex}%x
                                {\normalfont\large\bfseries}}
\renewcommand{\subsection}{\@startsection{subsection}{2}{0mm}%
                                {-1explus -.5ex minus -.2ex}%
                                {0.5ex plus .2ex}%
                                {\normalfont\normalsize\bfseries}}
\renewcommand{\subsubsection}{\@startsection{subsubsection}{3}{0mm}%
                                {-1ex plus -.5ex minus -.2ex}%
                                {1ex plus .2ex}%
                                {\normalfont\small\bfseries}}
\makeatother
\setcounter{secnumdepth}{0}
\setlength{\parindent}{0pt}
\setlength{\parskip}{0pt plus 0.5ex}
% -----------------------------------------------------------------------

\begin{document}
\raggedright
\footnotesize
\begin{center}
     \Large{\textbf{Elementi di Analisi 1 }} \\
\end{center}
\begin{multicols}{3}
\setlength{\premulticols}{1pt}
\setlength{\postmulticols}{1pt}
\setlength{\multicolsep}{1pt}
\setlength{\columnsep}{2pt}

\section{Goniometria}
\subsection{Identità basilari}
\begin{flalign*}
    & \sin^2(x) + \cos^2(x) = 1 &\\
    & \tan(x) = \frac{\sin(x)}{\cos(x)} &\\
    & \sec(x) = \frac{1}{\cos(x)} &\\
    & \csc(x) = \frac{1}{\sin(x)} &\\
    & \cot(x) = \frac{1}{\tan(x)} = \frac{\cos(x)}{\sin(x)} &\\
    & \sin(-x) = -\sin(x) &\\
    & \cos(-x) = \cos(x) &\\
\end{flalign*}

\subsection{Addizione e sottrazione}
\begin{flalign*}
    & \sin(x \pm y) = \sin(x)\cos(y) \pm \cos(x)\sin(y) &\\
    & \cos(x \pm y) = \cos(x)\cos(y) \mp \sin(x)\sin(y)
\end{flalign*}

\subsection{Duplicazione e bisezione}
\begin{flalign*}
    & \sin(2x) = 2 \sin(x) \cos(x) &\\
    & \cos(2x) = \cos^2(x) - \sin^2(x) &\\
    & \sin^2(x) = \frac{1 - cos(2x)}{2} &\\
    & \cos^2(x) = \frac{1 + cos(2x)}{2} &\\
    & \tan^2(x) = \frac{1 - cos(2x)}{1 + cos(2x)}
\end{flalign*}

\section{Limiti}
\subsection{Proprietà}
\begin{flalign*}
    & \lim_{{x \to a}} c = c, \quad \lim_{{x \to a}} x = a, \quad \lim_{{x \to a}} x^n = a^n &\\
    & \lim_{{x \to a}} [f(x) + g(x)] = \lim_{{x \to a}} f(x) + \lim_{{x \to a}} g(x) &\\
    & \lim_{{x \to a}} [f(x) \cdot g(x)] = \lim_{{x \to a}} f(x) \cdot \lim_{{x \to a}} g(x) &\\
    & \lim_{{x \to a}} \frac{f(x)}{g(x)} = \frac{\lim_{{x \to a}} f(x)}{\lim_{{x \to a}} g(x)}, \quad \text{if} \lim_{{x \to a}} g(x) \neq 0 &\\
    & \lim_{{x \to a}} [f(g(x))] = \lim_{{x \to a}} f(g(x)) &\\
    & \lim_{{x \to a}} f(x)^{g(x)} = \lim_{{x \to a}} e^{\ln [f(x)] g(x)} &\\
\end{flalign*}
\subsection{Limiti notevoli}
\vspace{1mm}
\begin{equation*}
    \begin{aligned}
        & \lim_{x \rightarrow 0} \frac{\ln(1+x)}{x} = 1 &\\
        & \lim_{x \rightarrow 0} \frac{\log_a(1+x)}{x} = \frac{1}{\ln(a)} &\\
        & \lim_{x \rightarrow 0} \frac{e^x - 1}{x} = 1 &\\
        & \lim_{x \rightarrow 0} \frac{a^x - 1}{x} = \ln(a) &\\
        & \lim_{x \rightarrow \pm \infty} \big(1 + \frac{\alpha}{x}\big)^x = e^{\alpha} &\\
        & \lim_{x \rightarrow 0} \frac{(1+x)^a - 1}{x} = a &\\
    \end{aligned}
    \hspace{12mm}
    \begin{aligned}
        & \lim_{x \rightarrow 0} \frac{sin(x)}{x} = 1 &\\
        & \lim_{x \rightarrow 0} \frac{1 - cos(x)}{x^2} = \frac{1}{2} &\\
        & \lim_{x \rightarrow 0} \frac{\tan(x)}{x} = 1 &\\
        & \lim_{x \rightarrow 0} \frac{\arcsin(x)}{x} = 1 &\\
        & \lim_{x \rightarrow 0} \frac{\arctan(x)}{x} = 1 &\\
    \end{aligned}
\end{equation*}
\vspace{1mm}
\subsection{Sviluppi di Taylor o Maclaurin}
\begin{align*}
    & e^x = 1 + x + \frac{x^2}{2!} + \frac{x^3}{3!} + \frac{x^4}{4!} + ... \hspace{8mm} \mbox{per } x \rightarrow 0 \\
    & \alpha^x = 1 + x \ln \alpha + \frac{x^2}{2!} \ln \alpha + \frac{x^3}{3!} \ln \alpha + \frac{x^4}{4!} \ln \alpha + ... \hspace{8mm} \mbox{per } x \rightarrow 0 \\
    & \log(1 + x) = x - \frac{x^2}{2} + \frac{x^3}{3} - \frac{x^4}{4} + \frac{x^5}{5} - ... \hspace{8mm} \mbox{per } x \rightarrow 0 \\
    & \sin(x) = x - \frac{x^3}{3!} - \frac{x^7}{7!} + ... \hspace{8mm} \mbox{per } x \rightarrow 0 \\
    & \cos(x) = 1 - \frac{x^2}{2!} + \frac{x^4}{4!} - \frac{x^6}{6!} + ... \hspace{8mm} \mbox{per } x \rightarrow 0 \\
    & \tan(x) = x + \frac{x^3}{3} + \frac{2x^5}{15} + o(x^6) \hspace{8mm} \mbox{per } x \rightarrow 0 \\
    & \arctan(x) = x - \frac{x^3}{3} + \frac{x^5}{5} - \frac{x^7}{7} + ...  \hspace{8mm} \mbox{per } x \rightarrow 0 \\
    & \arcsin(x) = x + \frac{x^3}{6} + \frac{3x^5}{40} + o(x^6)  \hspace{8mm} \mbox{per } x \rightarrow 0 \\
    & (1 + x)^{\alpha} = 1 + \alpha x + \frac{\alpha (\alpha - 1)}{2}x^2 + \frac{\alpha (\alpha - 1)(\alpha - 2)}{6}x^3 + o(x^3) \hspace{8mm} \mbox{per } x \rightarrow 0 \\
    & \sqrt{1 + x} = (1+x)^{\frac{1}{2}} = 1 + \frac{x}{2} - \frac{x^2}{8} + \frac{x^3}{16} + o(x^3)  \hspace{8mm} \mbox{per } x \rightarrow 0 \\
\end{align*}


 
\section{Numeri complessi} 
\begin{flalign*}
    & z = a + bi, \quad a, b \in \mathbb{R}, \quad i^2 = -1 \hspace{16mm} \mbox{ Definizioni di base} &\\
    & |z| = \sqrt{a^2 + b^2} \hspace{38.5mm} \mbox{ Modulo} &\\
    & \bar{z} = a - bi \hspace{45mm}  \mbox{ Coniugato} &\\
    & z = r(\cos \theta + i \sin \theta), \quad r = |z|, \quad \theta = \text{arg}(z) \hspace{1.5mm}  \mbox{ Forma polare} &\\
    & re^{i\theta} = r(\cos \theta + i \sin \theta) \hspace{28.5mm}  \mbox{ Formula di Eulero} &\\
    & (r(\cos \theta + i \sin \theta))^n = r^n (\cos (n\theta) + i \sin (n\theta)) \mbox{ Teorema di De Moivre} &\\
    & r = \sqrt{a^2 + b^2}, \quad \theta = \arctan\left(\frac{b}{a}\right) &\\
\end{flalign*}

\vfill
\hrule
\smallskip

\noindent Simone Calabrese, Università di Catania, A.A. 2022/2023 \hfill  \href{http://simonecalabrese.com/}{\color{blue}{simonecalabrese.com}}\\

\end{multicols}
\end{document}
